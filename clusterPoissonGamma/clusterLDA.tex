\title{Poisson Tucker Decomposition version of \\
	the Interaction-pattern Partitioned Topic Model}
%\author{
%        Vitaly Surazhsky \\
%                Department of Computer Science\\
%        Technion---Israel Institute of Technology\\
%        Technion City, Haifa 32000, \underline{Israel}
%            \and
%        Yossi Gil\\
%        Department of Computer Science\\
%        Technion---Israel Institute of Technology\\
%        Technion City, Haifa 32000, \underline{Israel}
%}
\author{Bomin Kim}
\date{\today}

\documentclass[11pt]{article}
\usepackage{microtype}
\usepackage{graphicx}
\usepackage{subfigure}
\usepackage{amsmath}
\usepackage{amssymb}
\usepackage{booktabs}
\usepackage{hyperref}
\begin{document}
\maketitle

%\begin{abstract}
%This is the paper's abstract \ldots
%\end{abstract}

\section{Generative Process}
To maintain single interaction pattern assignments (instead of admixture form which adds huge complexity in network history calculations), we assume an interaction-pattern distribution over $C$ unique interaction patterns
\begin{equation}
\boldsymbol{\psi}\sim \mbox{Dirichlet}\Big(\zeta, (\frac{1}{C},\ldots,\frac{1}{C})\Big),
\end{equation}
where $\zeta$ is the concentration parameter, and then each document $d \in [D]$ draws an interaction pattern $c_d$ as below:
\begin{equation}
c_d\sim \mbox{Multinomial}(\boldsymbol{\psi}).
\end{equation}
Next, we model the contents using Poisson Tucker Decomposition of Schein et al. (2016). First, each interaction pattern $c \in [C]$ has the IP-specific topic distribution
\begin{equation}
\theta_{ck} \sim \mbox{Gamma}(a_c,b_c),
\end{equation}
and each topic $k\in[K]$ has the topic-word distribution
\begin{equation}
\phi_{kv} \sim \mbox{Gamma}(\epsilon_0,\epsilon_0).
\end{equation}
Then, the number of tokens of type $v$ in document $d$ is
\begin{equation}
w_{dv} \sim \mbox{Poisson}(\sum_{k=1}^K \theta_{c_dk}\phi_{kv}),
\end{equation}
Therefore, $\boldsymbol{w}_d = (w_{d1},\ldots,w_{dV})$ is a very sparse vector with sum($\boldsymbol{w}_d$) = $N_d$. \\ \\
For tie generating process, we use the current version of the IPTM. For every possible author--recipient pair $(a,r)_{a \neq r}$, we define the ``recipient intensity", which is the likelihood of document $d$ being sent from $a$ to $r$:
\begin{equation}
\lambda_{adr} = {\boldsymbol{b}_{c_d}}^{\top}\boldsymbol{x}_{adrc_d},
\end{equation}
where we place a Normal prior $\boldsymbol{b}_c \sim N(\boldsymbol{\mu}_b, \Sigma_b)$. Similarly, we hypothesize ``If $a$ were the author of document $d$, when would it be sent?" and define the ``timing rate" for author $i$
\begin{equation}
\mu_{ad} = g^{-1}(\boldsymbol{\eta}_{c_d}^\top \boldsymbol{w}_{adc_d}),
\end{equation}
 with a Normal prior $\boldsymbol{\eta}_c \sim N(\boldsymbol{\mu}_\eta,\Sigma_\eta)$. We then follow the generalized linear model framework:
\begin{equation}
\begin{aligned}
E(\tau_{ad}) &= \mu_{ad},\\
V(\tau_{ad}) &= V(\mu_{ad}).
\end{aligned}
\end{equation}
\section{Derivation}
We first derive the sampling equation of $\theta$ and $\phi$, respectively. 
\begin{equation}
\theta_{ck}| \mbox{rest} \sim \mbox{Gamma}(a_c + \boldsymbol{w}_{ck}, b_c + \sum_{d: c_d = c}\sum_{v=1}^V\phi_{kv}),
\end{equation} 
where $\boldsymbol{w}_{ck} = \sum\limits_{d:c_d=c} \sum_{v=1}^V w_{dkv}$ with $w_{dkv} \sim \mbox{Multinomial}(w_{dv}, \theta_{c_dk}\phi_{kv})$.
\begin{equation}
\phi_{kv}| \mbox{rest} \sim \mbox{Gamma}(\epsilon_0 + \boldsymbol{w}_{kv}, \epsilon_0 +\sum_{d=1}^D\theta_{c_dk} ),
\end{equation} 
where $\boldsymbol{w}_{kv} = \sum_{d=1}^D w_{dkv}$ with $w_{dkv} \sim \mbox{Multinomial}(w_{dv}, \theta_{c_dk}\phi_{kv})$.\\ \newline
Since $u_{adr}$ is a binary random variable, new values may be sampled directly using
\begin{equation}
\begin{aligned}
&P(u_{adr}=1| \boldsymbol{u}_{ad\backslash r}, \boldsymbol{c},\boldsymbol{b}, \delta, \boldsymbol{x})
\propto \mbox{exp}\{\delta+\lambda_{adr}\};\\
&P(u_{adr}=0| \boldsymbol{u}_{ad\backslash r}, \boldsymbol{c},\boldsymbol{b}, \delta, \boldsymbol{x})\propto \text{I}(\lVert\boldsymbol{u}_{ad\backslash r}\rVert_1 > 0 ),
\end{aligned}
\label{eqn:latentreceiver}
\end{equation}
where $I(\cdot)$ is the indicator function that is used to prevent from the instances where the author has no recipients to send the document. \\\newline
New values for continuous variables $\delta, \boldsymbol{b},$ and $\boldsymbol{\eta}$ and $\sigma^2_\tau$ (if applicable) cannot be sampled directly from their conditional posteriors, but may instead be obtained using the Metropolis--Hastings algorithm. With uninformative priors (i.e., $N({0},\infty)$), the conditional posterior over $\delta$ and $\boldsymbol{b}$ is
\begin{equation}
\prod_{d=1}^D
\prod_{a=1}^A \frac{\exp\Big\{\mbox{log}\big(\text{I}( \lVert \boldsymbol{u}_{ad}\rVert_1 > 0)\big) + \sum\limits_{r \neq a} (\delta+\lambda_{adr})u_{adr}\Big\}}{Z(\delta,\boldsymbol{\lambda}_{ad})},
\end{equation}
where the two variables share the conditional posterior and thus can be jointly sampled. Likewise, assuming uninformative priors on $\boldsymbol{\eta}$ (i.e., $N({0},\infty)$) and $\sigma_{\tau}^2$ (i.e., half-Cauchy($\infty$)), the conditional posterior is
\begin{equation}
\prod_{d=1}^D\Big(\varphi_{\tau}(\tau_{d}; \mu_{a_d d}, \sigma_\tau^2)\times \prod_{a\neq a_d}\big(1-\Phi_{\tau}(\tau_{d}; \mu_{a d}, \sigma_\tau^2) \big)\Big).
\end{equation}
Finally, for each document $d \in [D]$, we sample interaction-pattern assignment from the discrete distribution over $C$ interaction patterns using 
\begin{equation}
\begin{aligned}
&P(c_d = c|\boldsymbol{\theta}, \zeta, \boldsymbol{u},\boldsymbol{a}, \boldsymbol{t})
\\&\propto P(c_d=c|\boldsymbol{c}_{\backslash d}, \zeta) P(\boldsymbol{\theta}_c|a_c, b_c, \boldsymbol{w}, c_d=c, \boldsymbol{c}_{\backslash d}, \boldsymbol{\theta}_{\backslash c})\\&\times P( \boldsymbol{w}_d|c_d = c, \boldsymbol{\theta}_c, \boldsymbol{\phi})\\
&\times P(a_d, t_d|c_d = c, \boldsymbol{\eta}, \sigma_\tau^2) P(\boldsymbol{u}|c_d=c, \boldsymbol{c}_{\backslash d}, \boldsymbol{b}, \delta) 
\\&\propto (\hat N_{c, \backslash d}+\frac{\zeta}{C}) \\&
\times \prod_{k=1}^K\mbox{dGamma}(\theta_{ck}; a_c + \boldsymbol{w}_{ck}, b_c +\sum\limits_{d:c_d=c} \sum_{v=1}^V\phi_{kv})\\&\times \prod_{v=1}^V \mbox{dPois}(w_{dv}; \sum_{k=1}^K \theta_{ck}\phi_{kv})\\&\times
\varphi_{\tau}(\tau_{d}; \mu_{a_d d}, \sigma_\tau^2)\times \prod_{a\neq a_d}\big(1-\Phi_{\tau}(\tau_{d}; \mu_{a d}, \sigma_\tau^2) \big)\\& 
\times
\prod_{a=1}^A \frac{\exp\Big\{\mbox{log}\big(\text{I}( \lVert \boldsymbol{u}_{ad}\rVert_1 > 0)\big) + \sum\limits_{r \neq a} (\delta+\lambda_{adr})u_{adr}\Big\}}{Z(\delta,\boldsymbol{\lambda}_{ad})}.\\
\end{aligned}
\end{equation}
\end{document}